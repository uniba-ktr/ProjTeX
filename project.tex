%===============================================================================
% Zweck:    KTR-Seminar-Vorlage
% Erstellt: 16.10.2007
% Updated:  27.06.2016
% Autor:    U.K. / M.G.
%===============================================================================

%===============================================================================
% Zum Kompilieren pdflatex und bibtex ausführen.
% Konfiguration in texmaker: Options -> Configure Texmaker -> Quick Build -> Select Latexmk + ViewPDF
%	Entsprechende Informationen in den config/metainfo verändern
% Zur Auswahl der Sprache im folgenden Befehl
% ngerman für deutsch eintragen, english für Englisch.
%===============================================================================

\documentclass[journal, onecolumn, a4paper, 12pt]{IEEEtran}
%===============================================================================
% zentrale Layout-Angaben und Befehle
%===============================================================================
\newcommand\meta{./meta}
\input{\meta/config/commands}
%===============================================================================
% LATEX-Dokument
%===============================================================================

\input{\meta/config/hyphenation}
\begin{document}
%===============================================================================
% Zum Kompilieren pdflatex und bibtex ausführen.
% Konfiguration in texmaker: Options -> Configure Texmaker -> Quick Build -> Select Latexmk + ViewPDF
% Entsprechende Informationen in den config/metainfo verändern
% Zur Auswahl der Sprache im folgenden Befehl
% ngerman für deutsch eintragen, english für Englisch.
%===============================================================================
\selectlanguage{english}

\maketitle

\pagenumbering{Roman}
\setcounter{page}{2}
%
\tableofcontents
% Einstellungen f\"{u}r Literaturverzeichnis
\newpage
\addcontentsline{toc}{section}{\listfigurename}
\listoffigures
\newpage
\addcontentsline{toc}{section}{\listtablename}
\listoftables
\newpage

% Glossar und Akronymverzeichnis, ungenutzte Verzeichnisse bitte auskommentieren
\printglossary[type=\acronymtype]  % list of acronyms
\newpage
\printglossary[type=symbolslist]   % list of symbols
\newpage
\printglossary[type=main]                     % main glossary
%===============================================================================
% LATEX-Dokument: Kapitel laden
%===============================================================================
%
\newpage
\pagenumbering{arabic}
\setcounter{page}{1}

%
% to use git tagging
%
\ifgit
  \input{\meta/exampleContent/version}
\fi
%
% hier einzelne Kapitel mit \input{Kapitel-File} einf\"{u}gen
%
\input{\meta/exampleContent/exampleContent}
%
%===============================================================================
% LATEX-Dokument: Literaturverzeichnis
%===============================================================================
%
\newpage
\phantomsection
% Einstellungen f\"{u}r Literaturverzeichnis
\addcontentsline{toc}{section}{\bibname}

\bibliographystyle{IEEEtran}
% argument is your BibTeX string definitions and bibliography database(s)
\bibliography{\meta/exampleLiterature/bib}
% Nutzung von Bibtex:
% hier den bib-file einbinden
%
% GATHER{bibfile.bib}
% \footnotesize
% \bibliography{bibfile}
% ansonsten: bbl als tex Datei einbinden
 %\input{KTR-Seminar-Literatur.tex}
%===============================================================================
% LATEX-Dokument: Literaturverzeichnis
%===============================================================================
\erklaerung
\end{document}
